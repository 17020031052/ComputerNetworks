% 环境:Windows10 + texstudio 2.12.16 + texlive2018;

\documentclass[UTF8]{article} % 这句限制了整个文档的类型
% 类型就是包括标题啊章节啊等等在内的一套设定
% 比如你想写一个类似论文的东西,就用article就好了
% 前面的UTF8是编码方式
% 注意 反斜杠\后面的是函数,函数值中暂时可以认为内容写在[中括号里],调整参数在{大括号里}
{\tiny }
\usepackage[zihao=-4]{ctex} % ctex可以写中文,中括号里那句的意思是正文小四号
\usepackage[a4paper]{geometry} % 调整纸张大小和页边距的包,这里中括号中规定了纸张大小
\geometry{left=2.0cm,right=2.0cm,top=2.0cm,bottom=2.0cm} % 页边距设置

\usepackage{graphicx} % 用它在报告里加图
\graphicspath{{image/}} % 指定图片所在文件夹

\usepackage{siunitx}

\begin{document}
	\vspace*{1cm}
	
	\begin{figure}[h]
		\centering
		\includegraphics[width=0.7\linewidth]{logo-1}
	\end{figure}

	\vspace*{0.5cm}
	
	\begin{center}
		\Huge{\textbf{计算机网络实验报告}}
		
		\Large{实验X}
	\end{center}
	
	\vspace*{0.5cm}
	
	\begin{table}[h]
		\centering	
		\begin{Large}
			\begin{tabular}{p{3cm} p{7cm}<{\centering}}
				姓\qquad 名: & 张三 \\
				\hline
				学\qquad 院: & 信息科学与工程学院 \\
				\hline
				学\qquad 号: & 00000000000 \\
				\hline
				分\qquad 组: & 第0组0号 \\
				\hline
				日\qquad 期: & 2019年9月23日 \\
				\hline
				指导教师: & 大表哥\\
				\hline
			\end{tabular}
		\end{Large}
	\end{table}

	\vspace*{1cm}
	
	%\textbf{摘要}\quad 本实验是个有趣的实验。
	\begin{abstract}
		本实验是个有趣的实验。
	\end{abstract}
	
	\newpage
	\tableofcontents
	\newpage
	
	\section{实验名称}
	植物大战僵尸游戏应如何通关。
	\section{实验目的}
	打通关!
	\section{实验内容}
	干就完事了。
	\section{实验步骤}
	\subsection{第一步}
	开机,打开游戏。
	\subsection{第二步}
	努力地玩
	\subsection{第三步}
	通关!
	\section{实验结果分析}
	没什么好说的。
	\section{心得体会}
	just so so。
	\section{参考资料}
	无
	
\end{document}