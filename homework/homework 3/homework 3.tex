% Computer Networks homework template.
% Environment: Windows10 Home + TeXLive2019 + Visual Studio Code.
\documentclass[a4paper,UTF8]{article}
\usepackage{ctex}
\ctexset{
proofname = \heiti{证明}
}
\usepackage{amsmath, amssymb, amsthm}
% amsmath: equation*, amssymb: mathbb, amsthm: proof
\usepackage{moreenum}
\usepackage{mathtools}
\usepackage{url}
\usepackage{bm}
\usepackage{enumitem}
\usepackage{graphicx}
\usepackage{subcaption}
\usepackage{booktabs} % toprule
\usepackage[mathcal]{eucal}
\usepackage[thehwcnt = 3]{iidef}

\thecourseinstitute{中国海洋大学}
\thecoursename{计算机网络}
\theterm{2019年秋季学期}
\hwname{作业}
\slname{\heiti{解}}
\begin{document}
\courseheader
\name{姓名:秦浩 \qquad 学号:17020031051}

\begin{enumerate}

\setlength{\itemsep}{3\parskip}

\item[3-07] 要发送的数据为$1101011011$.采用$CRC$的生成多项式是$P(X) = X^4 + X + 1$.试求应添加在数据后面的余数。\\
数据在传输过程中最后一个1变成了0,问接收端能否发现?\\
若数据在传输过程中最后两个1都变成了0,问接收端能否发现?\\
采用CRC检验后,数据链路层的传输是否就变成了可靠的传输?
\begin{solution}

\end{solution}

\item[3-08] 要发送的数据为$101110$.采用$CRC$的生成多项式是$P(X) = X^3 + 1$.试求应添加在数据后面的余数。
\begin{solution}

\end{solution}

\item[3-09]  一个$PPP$帧的数据部分(用十六进制写出)是7D\ 5E\ FE\ 27\ 7D\ 5D\ 7D\ 5D\ 65\ 7D\ 5E.试问真正的数据是什么(用十六进制写出)?
\begin{solution}

\end{solution}

\item[3-10] $PPP$协议使用同步传输技术传送比特串$0110111111111100$.试问经过零比特填充后变成怎样的比特串?若接收端收到的$PPP$帧的数据部分是$0001110111110111110110$,问删除发送端加入的零比特后变成怎样的比特串?
\begin{solution}

\end{solution}

\item[3-16] 数据率为$10Mbit/s$的以太网在物理媒体上的码元传输速率是多少码元/秒?
\begin{solution}

\end{solution}

\item[3-18] 试说明$10BASE-T$中的“$10$”、“$BASE$”和“$T$”所代表的意思。
\begin{solution}

\end{solution}

\item[3-20] 假定$1km$长的$CSMA/CD$网络的数据率为$1Gbit/s$。设信号在网络上的传播速率为$200000km/s$。求能够使用此协议的最短帧长。
\begin{solution}

\end{solution}

\item[3-21] 什么叫做比特时间?使用这种时间单位有什么好处?$100$比特时间是多少微秒?
\begin{solution}

\end{solution}

\item[3-22] 假定在使用$CSMA/CD$协议的$10Mbit/s$以太网中某个站在发送数据时检测到碰撞,执行退避算法时选择了随机数$r = 100$.试问这个站需要等待多长时间后才能再次发送数据?如果是$100Mbit/s$的以太网呢?
\begin{solution}

\end{solution}

\item[3-24] 假定站点A和B在同一个$10Mb/s$以太网网段上。这两个站点之间的传播时延为$225$比特时间。现假定A开始发送一帧,并且在A发送结束之前B也发送一帧。如果A发送的是以太网所容许的最短的帧,那么A在检测到和B发生碰撞之前能否把自己的数据发送完毕?换言之,如果A在发送完毕之前并没有检测到碰撞,哪么能否肯定A所发送到帧不会和B发送的帧发生碰撞?(提示:在计算时应当考虑到每一个以太网帧在发送到信道上时,在MAC帧前面还要增加若干字节的前同步码和帧定界符) 
\begin{solution}

\end{solution}

\item[3-25] 在上题中的站点A和B在$t=0$时同时发送了数据帧.当$t=255$比特时间,A和B同时检测到发生了碰撞,并且在$t=255+48=273$比特时间完成了干扰信号的传输.A和B在$CSMA/CD$算法中选择不同的r值退避.假定A和B选择的随机数分别是$r_A=0$和$r_B=1$.试问A和B各在什么时间开始重传其数据帧?A重传的数据帧在什么时间到达B?A重传的数据会不会和B重传的数据再次发生碰撞?B会不会在预定的重传时间停止发送数据?
\begin{solution}

\end{solution}

\item[3-26] 以太网上只有两个站,它们同时发送数据,产生了碰撞。于是按截断二进制指数退避算法进行重传。重传次数记为$i,i=1,2,3$,试计算第1次重传失败的概率第2次重传的概率、第3次重传失败的概率,以及一个站成功发送数据之前的平均重传次数I。 
\begin{solution}

\end{solution}

\item[3-27] 有10个站连接到以太网上,试计算以下三种情况下每一个站所能得到带宽。\\
(1)10个站点连接到一个10Mbit/s以太网集线器;\\
(2)10站点连接到一个100Mbit/s以太网集线器;\\
(3)10个站点连接到一个10Mbit/s以太网交换机。
\begin{solution}

\end{solution}

\item[3-28] $10Mbit/s$以太网升级到$100Mbit/s$、$1Gbit/s$和$10Gbit/s$时,都需要解决哪些技术问题?为什么以太网能够在发展的过程中淘汰掉自己的竞争对手,并使自己的应用范围从局域网一直扩展到城域网和广域网?
\begin{solution}

\end{solution}

\item[3-30] 在图3-30中,某学院的以太网交换机有三个接口分别和学院三个系的以太网相连。另外三个接口分别和电子邮件服务器、万维网服务器,以及一个连接互联网的路由器相连。图中的A,B和C都是$100Mbit/s$以太网交换机。假定所有的链路的速率都是$100Mbit/s$,并且图中的9台主机中的任何一个都可以和任何一个服务器或主机通信。试计算这9台主机和两个服务器产生的总的吞吐量的最大值,为什么? 
\begin{solution}

\end{solution}

\item[3-31] 假定在图3-30中的所有链路的速率仍然为$100Mbit/s$,但三个系的以太网交换机都换成为$100Mbit/s$的集线器。试计算这9台主机和两个服务器产生的总的吞吐量的最大值。为什么? 
\begin{solution}

\end{solution}

\item[3-32] 假定在图3-30中的所有链路的速率仍然为$100Mbit/s$,但所有的以太网交换机都换成为$100Mbit/s$的集线器。试计算这9台主机和两个服务器产生的总的吞吐量的最大值。为什么? 
\begin{solution}

\end{solution}

\item[3-33] 在图3-31中,以太网交换机有6个接口,分别接到5台主机和一个路由器。在下面表的“动作”一栏中,表示先后发送了4个帧。假定在开始时,以太网交换机的交换表是空的。试把该表中其他的栏目都填写完。

\end{enumerate}
\end{document}
\begin{equation}
\end{equation}